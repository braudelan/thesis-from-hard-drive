\section{Preliminary incubation}
    The preliminary incubation was designed to examine the possible effect of two LTTs, ORG and MIN, on SOM properties during a one week incubation, using two different amendments with differing lability.
    
    \subsection{Dynamics of SOM properties in non-amended samples}
    
        \subsubsection{Microbial biomass and respiration}
            both LTTs had seen high Resp rates (Fig \ref{fig:resp_control_preliminary}) in the first 24 h of incubation, with peak rates during the first hours of incubation, quickly decreasing to reach steady values of less than 20 mg CO2-C/kg/day from 48 h onward. A pulse of Resp was observed in Min, peaking at 74  mg CO2-C/kg/day after 7 h and a significantly smaller peak of 57 mg CO2-C/kg/day was concurrently recorded for Org. interestingly, Resp rates were lower in Org during the first 96h of incubation. It is possible that peak Resp for Org had occurred at some point between the 2h and 7h  sampling, thus being undetected. Our main experiment indeed showed peak Resp after 4h in control samples (results in following chapter). Still there is little reason to believe that an earlier peak in Resp would have been exclusively observed in Org as these respective LTTs presented otherwise similar patterns.
            following from Resp rate data, cumulative respiration was significantly higher in Min throughout most of the incubation, with a ~23$\%$ average increase over Org cumulative Resp values throughout the incubation
            A slight and evidently statistically insignificant, increase in MBC (Fig \ref{fig: mbc_ntrol_preliminary}) was observed in Min samples after 24 h of incubation, followed by slight decrease in the next 3 days.  In sharp contrast, in Org samples, MBC had seen a substantial increase of ~100 mg/kg between 24-48h of incubation. This increase marked a peak of MBC in Org, after which MBC levels declined sharply to reach a value of ~370 mg/kg, practically the same as in Min.
        \begin{figure}[H]
            \centering
            \includegraphics[scale=0.8]{figures/preliminary/absolute_values/Resp_CON.pdf}
            \caption{dynamics of $CO_2$ respiration in non-amended samples}
            \label{fig:resp_control_preliminary}
        \end{figure}
        
        \begin{figure}[H]
            \centering
            \includegraphics[scale=0.8, ]{images/preliminary/absolute/MBC_CON.pdf}
            \caption{dynamics of MBC in non-amended samples}
            \label{fig:mbc_control_preliminary}
        \end{figure}
        
        \subsubsection{WEOC}
            
            WEOC  (fig #) was higher in Org throughout the incubation, despite very similar values at 48h. Initial levels of WEOC were more than 50$\%$ higher in ORG, with this percentage decreasing to ~30$\%$ by 96 h of incubation. In both LTTs, WEOC was almost without change in the first 24 h, subsequently dropping sharply to reach a relatively similar value of ~20 mg/kg in both LTTs after 48 h of incubation. This sharp decrease was followed by a substantial increase in the next 48 h, finally reaching values  slightly, yet significantly higher than initial values. It is worth noting the opposite trends between MBC and WEOC, particularly in  Org, whereby  a sharp decrease in WEOC between 24-48 h was accompanied by a similarly sharp increase and the same (inverse) contrast was observed in the following 48 h.
    
            \begin{figure}[H]
            \centering
            \includegraphics[scale=0.8, ]{images/preliminary/absolute/WEOC_CON.pdf}
            \caption{dynamics of WEOC in non-amended samples}
            \label{fig:weoc_control_preliminary}
             \end{figure}
      
      
       \subsubsection{HWE total Carbon and Carbohydrate-C}
            
            Org had sustained significantly higher levels of HWEC (Fig \ref{fig:hwec_control_preliminary}) and HWES-C (Fig \ref{fig:hwes-c_control_preliminary}) compared with Min throughout the entire incubation, with Org values 50 and 30$\%$ higher than Min, for HWEC and HWES respectivly. HWEC and HWES-C had followed very similar patterns throughout the incubation, in both LTTs, while the dynamics of HWEC fluctuated more sharply in the first 96h in Org but not so much in Min which presented strong concurrence in time dependent changes between HWEC and HWES-C (pearson r = 0.97 for first differences in Min, compared with 0.91 for Org). These concurrent dynamics suggest that HWES-C comprised a relatively constant fraction of HWEC in control samples of these two LTT, throughout the incubation period. This fraction, calculated by averaging the ratio of HWES-C-to-HWEC across all sampling events, yielded a mean of 0.33 and 0.29 (assuming 40$\%$ carbon by weight in HWES) in Min and Org respectively, with a $\pm$0.01 associated error.   
        
            \begin{figure}[H]
                \centering
                \includegraphics[scale=0.8]{images/preliminary/absolute/HWEC_CON.pdf}
                \caption{dynamics of HWEC in non-amended samples}
                \label{fig:hwec_control_preliminary}
            \end{figure}
            
            \begin{figure}[H]
                \centering
                \includegraphics[scale=0.8]{images/preliminary/absolute/HWES-C_CON.pdf}
                \caption{dynamics of HWES-C in non-amended samples}
                \label{fig:hwes-c_control_preliminary}
            \end{figure}
       
        \subsubsection{Ergosterol}
            
            Min presented a  very slight decrease in Erg during the incubation period, with a mean Erg concentration of ~9 mg/kg, while Org sustained a considerable reduction in Erg concentrations from ~13 mg/kg in the first 24 h to \midtilde{}10.5 mg/kg by the end of the incubation (Fig. fig:erg_control_preliminary).
            
            \begin{figure}[H]
                \centering
                \includegraphics[scale=0.8]{images/preliminary/absolute/Erg_CON.pdf}
                \caption{dynamics of Ergosterol in non-amended samples}
                \label{fig:erg_control_preliminary}
            \end{figure}

    
    \subsection{Dynamics of SOM properties in Straw or KW Compost amended samples}
        
        \subsubsection{MB Carbon and Respiration}
            
            Org had seen an initial increase of \midtilde{}200 and \midtilde{}400 mg/kg MBC over control samples, following the addition of both Str and KWC (figures # and #) respectively, while for Min a smaller increase was observed after Str application (fig #). In Min, data is missing for the first sampling of KWC  amended samples (fig #). Nonetheless, the otherwise parallel MBC trends between the two LTTs ( in KWC amended samples) throughout the rest of the incubation, suggest a substantial initial increase of ~400 mg/kg over control samples (absolute value of 600 mg/kg MBC at the first sampling)  in Min+STR samples. 
            Despite initial increases in both STTs, subsequent dynamics differed notably. In KWC amended samples, the initial MBC increase (assuming 400 mg/kg control normalized MBC for Min samples) was followed by a general decrease in both LTTs, suggesting the favorable effect of KWC on MBC,  was short lived, at least in the short-term period recorded in this incubation. Org+KWC had seen a decrease in control normalized MBC from the intial 382 mg/kg to 250 mg/kg after 96h of incubation, while Min saw a similar or greater decrease across 4 days of incubation finally reaching a control normalized MBC value of  110 mg/kg. Moreover, 24 h after the start of the incubation, MBC in Min+KWC samples had decreased by almost 100 mg/kg below the level of corresponding control samples,  suggesting a short-term negative effect of KWC on MBC. In contrast with KWC, STR amendment had initially caused relatively small and little-to-no increases over control samples in Org and Min respectively, while seeing a sharp increase of 600 mg/kg in both LTTs (~50$\%$ higher than initial increase in KWC amended samples). This increase was subsequently leveled off in the next 4 days. 
            Resp dynamics in the two STTs seem to concur with their respective MBC dynamics. KWC Resp rates peaked immediately at the onset of the incubation and subsequently decreased rapidly, reaching control normalized values of close to 20 mg CO2-C /kg/day after 48 hours in both LTTs, while STR Resp rates peaked 7 and 10 h after incubation began, in Min and Org respectively. Indeed Resp data in KWC is missing for the 7h (as well as 24 h) sampling event, so that it is possible that peak rates would have been detected after 7h of incubation. Nonetheless, the substantial difference of more than 150 mg/kg (~75$\%$ ) between the 2h and 10h sampling suggest that, even if this was true, this peak would probably not have been much higher than the observed peak. This intense initial respiration response followed by rapid decline towards control respiration rates, corresponds  to an initial MBC increase in these KWC amended samples. Similarly, a somewhat delayed Resp peak in STR amended samples and relatively high Resp rates sustained throughout the larger part of the incubation, concur with substantial growth in the first 24 h of incubation as well as with the high levels of MBC sustained through the next 72 h in these samples.      

    
\section{Effect of management history on SOM properties in non-amended samples}
   
