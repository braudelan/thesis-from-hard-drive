% !TeX root = ../main.tex


The  main goal of this work was to examine the effect of long term, contrasting management treatments, on different \gls{som} pools. This effect was considered from two aspects, (1) the effect on baseline properties (i.e normal fluctuations of non-amended samples averaged across a 28 days period) and (2) the effect of labile organic input on short term  carbon dynamics, particularly with regard to variations in microbial carbon use efficiency.\\
My results showed substantial increases in various \gls{som} features as well as in an AS test, for field crops cultivated soil, particularly when combined with organic fertilization, when compared with a non-cultivated, unmanaged soil. This result was unexpected based on the majority of literature on the subject. The specific climatic conditions in the area are recognized as a considerable factor involved in this unusual outcome, as many of the works on this subject have been carried out in more humid climates. it was postulated that increased plant activity in the cultivated compared with non-cultivated soils, a phenomenon that is mostly restricted to arid and semi-arid climates, was primarily responsible for the observed enhancement of \gls{som} pools in the cultivated soils. This finding is in line with the new paradigim of \gls{som} buildup. \\
Management effect was not as clear when the contrasting, organic and mineral fertilization treatments where compared. This was not altogether unexpected given the prior results from the same long term experimental plots but also considering the relative similarity between these two long-term treatments, especially during the last two years of the GOP experiment before the soil was sampled for the current experiment. Nonetheless, considerable differences were indeed observed between these two contrasting fertilization treatments, showing, on the whole, a definite increase in \gls{som} pools and AS after 5 years of intense organic fertilization when compared with mineral-only fertilization. That this distinction was observed in soil samples obtained after a substantial resting period (\textit{no input + fallow}) in the original long-term experiment (GOP) is of particular interest, as it expresses the lasting effect of compost amendment.\\
The introduction of labile organic input to the soils revealed considerable variation between the three long-term treatments, particularly with regard to microbial $ CO_2 $ release and the consumption of labile carbon fraction. The dynamics of WEOC and Resp following MRE additions presented clear differences between the three LTTs in the efficiency by which easily available organic carbon was removed from the Water Extractable Organic Matter pool in the short term (hours-days). This efficiency was linked with the soil's microbial metabolic capacity as it corresponded with higher extant MBC , basal Resp and \gls{som} stocks. Additionally, the comparison of WEOC trends with those of microbial respiration clearly demonstrated the role of DOM fractions such as WEOC as a readily available source of microbial consumption. In contrast, no clear differences between LTT's were observed when a weekly balance of microbial growth and cumulative respiration was used to calculate \gls{cue}. Therefore, it remains unclear whether and in what way the efficiency of short-term labile carbon removal, particularly under the stress of an intense load of labile substrate, was ultimately related to \gls{cue}.\\
My hypothesis stated that the effect of LTTs would be expressed in microbial \gls{cue}, thus providing practical indications as to preferred management for enhancing \gls{som} accrual. The fact that my \gls{cue} calculations did not reveal any clear trend regarding the effect of  LTTs on microbial substrate utilization, emerges as a significant limitation in this work. The ability to track precisely and quantitatively the flow of substrate, as well as soil carbon into and through different \gls{som} pools (e.g. by carbon labeling) would certainly have been highly beneficial in this respect. Using this kind of carbon tracing methods, possibly combined with more sensitive and frequent sampling of MBC dynamics, may have allowed a more accurate evaluation of \gls{cue}.\\
High load substrate application in the MRE incubation had evidently initiated an intense soil response and this effect may have overshadowed the effect of management history. Thus, investigation in this direction would probably require the use of smaller amounts of substrate. \\
Another apparent drawback for the current work is the relatively small sample of long-term treatments (N=3) and the lack of geographical and basic mineralogical variation between these soil samples. This limited the ability to detect meaningful correlations between the different measured parameters and possibly find different mechanistic explanations. Although not explicitly included in the research objectives, I nonetheless tried to provide some sensible mechanistic interpretations to the outcomes of this work. The conceptual model (section \ref{stabilizaton_model}), suggested in the context of short term variations observed in CUE, could well have been used to explore the interactions of soil features like microbial activity and aggregate structure, with the short term processes of substrate decomposition and potential stabilization. A broader spectrum of soils with differing initial statuses, would have been highly instrumental for this cause. 

A secondary objective for this work, was to evaluate the effect of consecutive applications of a highly labile substrate (MRE), on short-term SOM dynamics, particularly with regard to microbial substrate use efficiency. WEOC dynamics and changes in CUE between consecutive weeks demonstrated the strong effect of these successive applications, on microbial utilization of substrate C and the ability to remove labile OM from the soluble fraction (WEOC). A conceptual model was envisioned, describing the principal drivers controlling the potential for carbon stabilization in this specific system. In light of this stabilization model, I proposed that the disproportion between microbial agents of decomposition and spatial sites of potential OM stabilization on the one hand and the increasing load of labile soluble OM on the other hand, resulted in a clear reduction in the overall CUE.
The scope of this work and methodological limitations did not permit a thorough examination and validation of this model. Nonetheless, I suggest that this simplistic and basic model, or a similar model along the same lines, can greatly benefit the assessment and study of the soil potential for carbon stabilization.     	 
